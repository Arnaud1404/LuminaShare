\documentclass[a4paper,12pt]{article}
\usepackage[frenchb]{babel}
\usepackage[utf8]{inputenc}
\usepackage[T1]{fontenc} 
\usepackage{lmodern}
\usepackage{mathptmx}

\usepackage{amsmath}
\usepackage{etoolbox}
\usepackage{float}
\usepackage{geometry}
\usepackage{hyperref}
\hypersetup{
    colorlinks=true,
    linkcolor=blue,
    filecolor=magenta,
    urlcolor=cyan,
    pdftitle={Projet de Développement Logiciel - Cahier des besoins},
    pdfpagemode=FullScreen,
    }
\usepackage{graphicx}
%\usepackage[disable]{todonotes}
\setlength{\marginparwidth}{2cm}
\usepackage{todonotes}
\usepackage{titlesec}
\titleformat*{\section}{\Large\bfseries\sffamily}
\titleformat*{\subsection}{\large\bfseries\sffamily}
\titleformat*{\subsubsection}{\itshape\subsubsectionfont}

\geometry{margin=2cm}

\newcounter{besoin}

% Descriptif des besoins:
% 1 - Label du besoin pour référencement
% 2 - Titre du besoin
% 3 - Description
% 4 - Gestion d'erreurs
% 5 - Spécifications tests
\newcommand{\besoin}[5]{%
  \refstepcounter{besoin}%
  \fbox{\parbox{0.95\linewidth}{%
    \begin{center}\label{besoin:#1}\textbf{\sffamily Besoin~\thebesoin~: #2}\end{center}
    \ifstrempty{#3}{}{\textbf{Description~:} #3\par}
    \vspace{0.5em}
    \ifstrempty{#4}{}{\textbf{Gestion d'erreurs~:} #4\par}
    \vspace{0.5em}
    \ifstrempty{#5}{}{\textbf{Tests~:} #5\par}
  }}
}

\newcommand{\refBesoin}[1]{%
  Besoin~\ref{besoin:#1}%
}

\title{\sffamily \textbf{LuminaShare}}
\author{Arnaud Gomes, Kamiel De Vos Le plus, Soraya Benachour}
\date{}

\begin{document}

\maketitle


\section{Introduction}

Ce document présente les besoins nécessaires au développement d'une application
de recherche d'image par similarité avec une architecture de type client-serveur s'appuyant sur une base de données pour l'indexation des images.
Afin de permettre aux groupes d'approfondir un sujet qui les intéresse plus
particulièrement, on propose la structure suivante.

\begin{description}
\item[Noyau commun :] Chaque groupe devra réaliser l'implémentation du fonctionnement central de l'application client-serveur. Elle est décrite dans la section~\ref{sec:kernel}. Notez qu'une partie des besoins a été développée dans les TP communs. Le code du noyau commun fera l'objet d'un rendu intermédiaire.
  
\item[Extensions :] Code couleur des extensions : 

\begin{itemize}
  \item \textcolor{cyan}{Bleu} : Un besoin qui a été modifié par rapport à sa version initiale
  \item \textcolor{orange}{Orange} : Un nouveau besoin qui n'a pas pu être implémenté par manque de temps
  \item \textcolor{red}{Rouge} : Un nouveau besoin ajouté au cahier 
\end{itemize}

\end{description}


Chaque groupe peut faire évoluer ce document avec l'aval de son chargé de TD. Le cahier des besoins fera partie des rendus.\\


L'application devra permettre de traiter les images couleur enregistrées aux formats suivants :

\begin{itemize}
\item \verb!JPEG!
\item \verb!PNG!
\end{itemize}



\section{\label{sec:kernel}Noyau commun}

\subsection{Serveur}

\besoin{server:initImages}
{\textcolor{cyan}{Initialiser un ensemble d'images présentes sur le serveur}}
{
 Lorsque le serveur est lancé, il doit vérifier l'existence du dossier \verb!images! à l'endroit où le serveur est exécuté. S'il n'existe pas, le serveur doit créer ce dossier automatiquement. Ensuite, il charge toutes les images présentes dans ce dossier et les indexe dans une base de données (cf. \refBesoin{server:indexImage}). Seuls les fichiers images correspondants aux formats reconnus (JPEG, PNG) doivent être traités. Les sous-dossiers du dossier \verb!images! ne seront pas traités.
}
{
Si la création du dossier \verb!images! échoue (par exemple, en raison de permissions insuffisantes), une erreur explicite doit être levée avec un message indiquant la cause.
}
{
  \begin{enumerate}
  \item Lancement du serveur dans un environnement où le dossier \verb!images! n'existe pas : vérifier que le dossier est créé automatiquement et que le serveur démarre sans erreur.
  \item Mise en place d'un dossier de test contenant des images aux formats reconnus ainsi que des documents avec des extensions non-reconnues comme étant des images (e.g.,~\verb!.txt!) et vérifier que seules les images JPEG et PNG sont indexées..
  \end{enumerate}
}

\besoin{server:manageImages}
{Gérer les images présentes sur le serveur}
{  
    Le serveur gère un ensemble d'images. Il permet d'accéder aux données brutes de chaque image ainsi qu'aux méta-données nécessaires aux réponses aux requêtes (identifiant, nom de fichier, taille de l'image, format,...). Le serveur peut :

    \begin{enumerate}
    \item accéder à une image via son identifiant,
    \item supprimer une image via son identifiant,
    \item ajouter une image,
    \item construire la liste des images disponibles (composée uniquement des métadonnées).
    \end{enumerate}

    Le serveur garantit la persistance des données : le dossier \verb!images! contient à tout moment l'ensemble des images disponibles qui sont également indexées dans la base de données.
}
{}
{}

\besoin{server:indexImage}
{Indexer une image}
{
  Le serveur permet d'indexer une image c'est-à-dire d'ajouter dans une base de données un enregistrement correspondant à l'image (identifiant et descripteur). Les descripteurs supportés sont l'histogramme 2D Teinte/Saturation et l'histogramme 3D RGB de l'image.
}
{}
{}

\besoin{server:searchBySimilarity}
{Rechercher des images similaires à une image donnée}
{
  Le serveur permet de construire la liste des \verb!N! images les plus similaires à une image donnée pour un descripteur donné.
}
{}
{}

\besoin{server:authentication}
{\textcolor{red}{Système d'authentification utilisateur}}
{
  Le serveur doit implémenter un système d'authentification basique avec identifiants uniques et mots de passe. Il stocke les données utilisateur de façon sécurisée dans la base de données :
  \begin{itemize}
    \item userid (identifiant unique choisi par l'utilisateur, clé primaire)
    \item name (pseudo affiché)
    \item bio (courte description de l'utilisateur)
    \item password (mot de passe chiffré)
  \end{itemize}
  Les mots de passe sont stockés de manière sécurisée à l'aide de l'algorithme SHA256, le mot de passe est encrypté en Base64 puis en SHA256.
}
{
  En cas d'échec d'authentification, le serveur renvoie un code d'erreur 401 (Unauthorized).
}
{}

\besoin{server:userManagement}
{Gestion des utilisateurs et de leurs contenus}
{
  Le serveur associe chaque image à son utilisateur créateur via une clé étrangère (userid). Les utilisateurs peuvent :
  \begin{itemize}
    \item {\textcolor{red}{Consulter leur profil et leurs images}}
    \item {\textcolor{orange}{Visualiser les profils des autres utilisateurs}}
    \item {\textcolor{red}{Gérer uniquement leurs propres images}}
  \end{itemize}
  Les images disposent d'un attribut "public" (booléen) qui détermine leur visibilité.
}
{}
{}

\besoin{server:extendedInfo}
{Informations enrichies pour les images}
{
  Chaque image stockée dans la base de données contient des informations supplémentaires :
  \begin{itemize}
    \item {\textcolor{orange}{userDescription}} (texte libre décrivant l'image)
    \item {\textcolor{orange}{tags}}  (liste de mots-clés associés à l'image)
    \item {\textcolor{red}{userid}} (identifiant de l'auteur de l'image)
    \item {\textcolor{red}{public}} (booléen indiquant si l'image est accessible à tous)
  \end{itemize}
}
{}
\newpage

\subsection{Communication}

Pour l'ensemble des besoins, des codes d'erreurs à renvoyer sont précisés dans le paragraphe "Gestion d'erreurs" (à compléter si nécessaire). \\


\besoin{comm:listImages}
{Transférer la liste des images existantes}
{
  La liste des images présentes sur le serveur doit être envoyée par le serveur lorsqu'il reçoit une requête \verb!GET! à l'adresse \verb!/images!.

  Le résultat sera fourni au format \verb!JSON!, sous la forme d'un tableau
  contenant pour chaque image un objet avec les informations suivantes :
  \begin{description}
  \item[Id:] L'identifiant auquel est accessible l'image (type \verb!long!)
  \item[Name:] Le nom du fichier qui a servi à construire l'image (type \verb!string!)
  \item[Type:] Le type de l'image (type \verb!org.springframework.http.MediaType!) 
  \item[Size:] Une description de la taille de l'image, par exemple \verb!640*480! pour
    une image de $640 \times 480$ pixels (type \verb!string!)
  \end{description}
}
{}
{
  Pour le dossier de tests spécifié dans \refBesoin{server:initImages}, la
  réponse attendue doit être comparée à la réponse reçue lors de l'exécution de la commande.
}

\besoin{comm:userRoutes}
{\textcolor{red}{Routes pour la gestion des utilisateurs}}
{
  Le serveur expose les routes suivantes pour la gestion des utilisateurs :
  \begin{itemize}
    \item \verb!/api/auth/login! (POST) : Authentification d'un utilisateur
    \item \verb!/api/auth/register! (POST) : Création d'un nouvel utilisateur
    \item \verb!/user/{userid}! (GET) : Récupération des informations d'un utilisateur
    \item \verb!/user/{userid}/images! (GET) : Récupération des images d'un utilisateur
  \end{itemize}
}
{
  \begin{description}
    \item[200 OK] : La requête a été traitée avec succès
    \item[201 Created] : L'utilisateur a été créé avec succès
    \item[400 Bad Request] : Format de données incorrect
    \item[401 Unauthorized] : Authentification échouée
    \item[404 Not Found] : Utilisateur non trouvé
  \end{description}
}
{}


\besoin{comm:create}
{Ajouter une image}
{
  L'envoi d'une requête \verb!POST! à l'adresse \verb!/images! avec
  des données de type multimedia dans le corps de la requête doit ajouter une
  image à celles disponibles sur le serveur (voir \refBesoin{server:manageImages}).
}
{
  \begin{description}
  \item[201 Created:] La requête s'est bien exécutée et l'image est à présent
    sur le serveur.
  \item[415 Unsupported Media Type:] La requête a été refusée car le serveur ne
    supporte pas le format reçu (par exemple \verb!EXR!).
  \end{description}
}
{}

\besoin{comm:retrieve}
{Récupérer une image}
{
  L'envoi d'une requête \verb!GET! à une adresse de la forme \verb!/images/id!
  doit renvoyer l'image stockée sur le serveur avec l'identifiant \verb!id! (entier positif). En cas de succès, l'image est retournée dans le corps de la réponse.
}
{
  \begin{description}
  \item[200 OK:] L'image a bien été récupérée.
  \item[404 Not Found:] Aucune image existante avec l'identifiant \verb!id!.
  \end{description}
}
{}


\besoin{comm:delete}
{Supprimer une image}
{
  L'envoi d'une requête \verb!DELETE! à une adresse de la forme \verb!/images/id!
  doit effacer l'image stockée avec l'identifiant \verb!id! (entier positif). Voir \refBesoin{server:manageImages}.
}
{
  \begin{description}
  \item[200 OK:] L'image a bien été effacée.
  \item[404 Not Found:] Aucune image existante avec l'identifiant \verb!id!.
  \end{description}
}
{}

\besoin{comm:getsimilar}
{Transférer la liste des images les plus similaires à une image donnée}
{
    Lors d'une requête \verb!GET! à une adresse de la forme \verb!/images/id/similar?number=N\&descriptor=DESCR! le serveur envoie la liste des \verb!N! images les plus similaires à l'image d'identifiant \verb!id! selon le descripteur \verb!DESCR!.

    Le résultat sera fourni au format \verb!JSON!, sous la forme d'un tableau
  contenant les informations de chaque image (comme pour le \refBesoin{comm:listImages})
}
{
  \begin{description}
  \item[200 OK:] La requête a bien été traitée.
  \item[400 Bad Request:] La requête ne peut être traitée, par exemple si la valeur du paramètre \verb!descriptor! ne correspond pas à un descripteur disponible.
  \item[404 Not Found:] Aucune image existante avec l'indice \verb!id!.
  \end{description}
}
{}


\newpage

\subsection{Client}
Les actions que peut effectuer l'utilisateur côté client induisent des requêtes envoyées au serveur. En cas d'échec d'une requête, le client doit afficher un message d'erreur explicatif.\\

\besoin{client:viewImages}
{Parcourir les images disponibles sur le serveur}
{
  L'utilisateur peut visualiser les images disponibles sur le serveur. La présentation visuelle peut prendre la forme d'un carroussel ou d'une galerie d'images. On suggère que chaque vignette contenant une image soit de taille fixe (relativement à la page affichée). Suivant la taille de l'image initiale la vignette sera complètement remplie en hauteur ou en largeur.
}
{}
{}

\besoin{client:authentication}
{\textcolor{red}{Interface d'authentification}}
{
  L'utilisateur peut s'inscrire et se connecter via des formulaires dédiés. Une fois connecté, il peut accéder à ses fonctionnalités privées.
}
{}
{}

\besoin{client:homePage}
{\textcolor{red}{Page d'accueil inspirée de Pinterest}}
{
  La page d'accueil présente une galerie d'images dans un style inspiré de Pinterest, avec un champ permettant de liker une image. Les images publiques de tous les utilisateurs y sont affichées.
}
{}
{}

\besoin{client:profilePage}
{\textcolor{red}{Page de profil utilisateur}}
{
  Chaque utilisateur dispose d'une page de profil affichant :
  \begin{itemize}
    \item Son identifiant(userid), pseudo (name) et sa biographie (bio)
    \item Une galerie de ses images publiques
    \item Pour le propriétaire du profil : toutes ses images (publiques et privées)
  \end{itemize}
  Le propriétaire du profil peut également ajouter, supprimer ou éditer ses images depuis cette page.
}
{}
{}

\besoin{client:search}
{\textcolor{orange}{Barre de recherche}}
{
  Une barre de recherche permet à l'utilisateur de filtrer les images par :
  \begin{itemize}
    \item Mots-clés (tags)
    \item Identifiant utilisateur (userid)
  \end{itemize}
}
{}
{}

\besoin{client:selectImage}
{Sélectionner une image et afficher les images similaires}
{
  L'utilisateur peut cliquer sur la vignette correspondant à une image. L'image est affichée sur la page. L'utilisateur peut visualiser les méta-données de l'image. L'utilisateur peut choisir d'afficher les images similaires disponibles. Il peut préciser le nombre d'images similaires à afficher et le descripteur utilisé pour la mesure de similarité. Pour chaque image similaire est affiché un score de similarité.
}
{}
{}


\besoin{client:saveImage}
{Enregistrer une image sur disque}
{
  L'utilisateur peut sauvegarder dans son système de fichier une image chargée.
}
{}
{}

\besoin{client:createImage}
{Ajouter une image aux images disponibles sur le serveur}
{
  L'utilisateur peut ajouter une image choisie dans son système de fichier aux images disponibles sur le serveur. Cet ajout est persistant (un fichier est ajouté côté serveur).
}
{}
{}



\besoin{client:delete}
{Supprimer une image}
{
  Le client peut choisir de supprimer une image préalablement sélectionnée. Elle n'apparaîtra plus dans les images disponibles sur le serveur. Cette suppression est persistante (un fichier est supprimé côté serveur).
}
{}
{}

\besoin{client:preview}
{\textcolor{red}{Prévisualiser l'application d'un filtre sur une image}}
{
  L'utilisateur peut sélectionner un filtre à appliquer à une image sélectionnée.\\
  Dans une séction prédéfinis, une image avec le filtre appliqué apparaîtra en temps réeel
}
{}
{}

\besoin{client:saveFilteredImage}
{\textcolor{orange}{Sauvegarder une image filtrée dans la galerie personnelle}}
{
  L'utilisateur peut sauvegarder une image après application d'un filtre (par exemple, redimensionnement, rotation) directement dans sa galerie personnelle sur le serveur. La nouvelle image conserve les mêmes paramètres que l'image originale, tels que la visibilité (publique ou privée). Le nom du fichier doit être unique pour éviter les conflits.
}
{
  Si l'enregistrement échoue (par exemple, en raison d'un problème de connexion ou d'un nom de fichier déjà existant), un message d'erreur explicatif doit être affiché à l'utilisateur.
}
{}

%% \newpage

%% \subsection{Traitement d'images}
%% \label{tai}

%% \besoin{tai:luminosity}
%% {Réglage de la luminosité}
%% {L'utilisateur peut augmenter ou diminuer la luminosité de l'image sélectionnée.}
%% {}
%% {}

%% \besoin{tai:equalizeHist}
%% {Égalisation d'histogramme}
%% {L'utilisateur peut appliquer une égalisation d'histogramme à l'image sélectionnée. L'égalisation sera apliquée au choix sur le canal S ou V de l'image représentée dans l'espace HSV.}
%% {}
%% {}

%% \besoin{tai:setHue}
%% {Filtre coloré}
%% {L'utilisateur peut choisir la teinte de tous les pixels de l'image sélectionnée de façon à obtenir un effet de filtre coloré.}
%% {}
%% {}

%% \besoin{tai:blur}
%% {Filtres de flou}
%% {L'utilisateur peut appliquer un flou à l'image sélectionnée. Il peut définir le filtre appliqué (moyen ou gaussien) et choisir le niveau de flou. La convolution est appliquée sur les trois canaux R, G et B.}
%% {}
%% {}

%% \besoin{tai:contour}
%% {Filtre de contour}
%% {L'utilisateur peut appliquer un détecteur de contour à l'image sélectionnée. Le résultat sera issu d'une convolution par le filtre de Sobel. La convolution sera appliquée sur la version en niveaux de gris de l'image.}
%% {}
%% {}

\newpage
\subsection{{\textcolor{red}{Traitement d'images}}}
\label{tai}

\besoin{tai:route}
{Routes pour le traitement d'images}
{Le serveur expose la route /images/\verb!id!/filter (GET) avec les paramètres suivants pour le traitement des images

% filter=${filter}&number=${number}&height=${height}
\begin{itemize}
  \item filter - le filtre à appliquer
  \item number - un argument demandé par le filtre
  \item height (optionnel) - Spécifique au filtre \verb!resize!, permet de redimensionner les images rectangles
\end{itemize}
}
{
  \begin{description}
    \item[200 OK:] La requête a bien été traitée.
    \item[400 Bad Request:] La requête ne peut être traitée, par exemple si la valeur du paramètre \verb!filter! est inconnu.
    \item[404 Not Found:] Aucune image existante avec l'indice \verb!id!.
  \end{description}

}
{}
\besoin{tai:resize}
{Redimensionnement des images}
{
  L'utilisateur peut redimensionner une image sélectionnée à une taille standard (des valeurs standard pour une hauteur et largeuer selon le beosin ou le choix) pour optimiser son stockage ou son affichage.
  La fonctionne redimensionne bien l'image selon les valeur qu'on propose mais la qualité d'image se diminue.
}
{
  Si l'image ne peut pas être redimensionnée (par exemple, en cas de corruption), un message d'erreur explicatif doit être affiché à l'utilisateur.
}
{}

\besoin{tai:invertColors}
{Inversion des couleurs}
{
  L'utilisateur peut inverser les couleurs d'une image sélectionnée pour obtenir un effet de négatif (par exemple, le blanc devient noir, le rouge devient cyan).
}
{
  Si l'image est invalide ou corrompue, un message d'erreur explicatif doit être affiché à l'utilisateur.
}
{}

\besoin{tai:mirror}
{Miroir de l'image}
{
  L'utilisateur peut créer une version miroir de l'image sélectionnée, soit horizontalement (inversion gauche-droite), soit verticalement (inversion haut-bas).
}
{
  Si l'image ne peut pas être traitée (par exemple, en cas de corruption), un message d'erreur explicatif doit être affiché à l'utilisateur.
}
{}

\besoin{tai:rotate}
{Rotation des images}
{
  L'utilisateur peut faire pivoter une image sélectionnée selon un angle spécifié (par exemple, 90°, 180°, ou 270°) pour corriger son orientation ou appliquer un effet visuel. 
}
{
  Si l'image ne peut pas être pivotée (par exemple, en cas de corruption ou d'angle invalide), un message d'erreur explicatif doit être affiché à l'utilisateur.
}
{}

\besoin{tai:luminosity}
{Réglage de la luminosité}
{L'utilisateur peut augmenter ou diminuer la luminosité de l'image sélectionnée.}
{}
{}

\besoin{tai:blur}
{Filtres de flou}
{L'utilisateur peut appliquer un flou à l'image sélectionnée. Il peut définir le filtre appliqué (moyen ou gaussien) et choisir le niveau de flou. La convolution est appliquée sur les trois canaux R, G et B.}
{}
{}

\newpage

\subsection{Besoins non-fonctionnels}

\besoin{bnf:containerization}
{\textcolor{red}{Conteneurisation avec Docker}}
{
  L'application sera conteneurisée à l'aide de Docker pour simplifier le déploiement. 
}
{}
{}

\besoin{bnf:internationalization}
{\textcolor{red}{Internationalisation}}
{
  L'application supportera le français, l'anglais et l'arabe . Les traductions seront stockées dans des fichiers JSON et l'utilisateur pourra
  basculer entre les langues via un sélecteur dans l'interface. L'interface doit prendre en charge l'écriture de droite à gauche (RTL) pour l'arabe afin d'assurer une expérience utilisateur correcte.
}
{}
{}

\besoin{bnf:responsiveDesign}
{\textcolor{red}{Design responsive}}
{Design responsive}
{
  L'interface client s'adaptera automatiquement aux différentes tailles d'écran, permettant une utilisation confortable sur ordinateur et portable.
}
{}
{}

\besoin{bnf:ci}
{Intégration continue}
{
  Les outils de Gitlab permettant le développement partagé et l'intégration continue seront utilisés pour :
  \begin{itemize}
    \item le versionnage,
    \item la gestion des tickets,
    \item la compilation de l'application,
    \item l'exécution des tests.
  \end{itemize}
  
}
{}
{}


\besoin{bnf:serverCompatibility}
{Compatibilité du serveur}
{
  La partie serveur de l'application sera écrite en Java (JDK 17) avec les
  bibliothèques suivantes:
  \begin{itemize}
  \item \verb!org.springframework.boot! : version 3.2.x
  \item \verb!org.boofcv! : version 1.1.2
  \item \verb!org.postgresql! : version 42.7.1
  \item \verb!com.pgvector!: version 0.1.4
  \end{itemize}

  La base de données utilisée sera PostgreSQL avec l'extension \verb!pgvector!. Spring JDBC sera utilisé pour faire le lien avec la base.\\

  Le fonctionnement du serveur devra être éprouvé sur au moins un des environnements suivants :
  \begin{itemize}
  \item Windows $\ge$ 10
  \item Ubuntu $\ge$ 20.04
  \item Debian Bookworm
  \item MacOS $\ge$ 11
  \end{itemize}
}
{}
{}

\besoin{bnf:clientCompatibility}
{Compatibilité du client}
{
  Le client sera écrit en TypeScript et s'appuiera sur la version \verb!3.x! du
  framework \verb!Vue.js!.\\

  Le client devra être testé sur au moins l'un des navigateurs Web suivants,
  la version à utiliser n'étant pas imposée :
  \begin{itemize}
  \item Safari
  \item Google Chrome
  \item Firefox
  \end{itemize}
}
{}
{}

\besoin{bnf:documentation}
{Documentation d'installation et de test}
{
  La racine du projet devra contenir un fichier \verb!README.md! indiquant au
  moins les informations suivantes:
  \begin{itemize}
  \item Système(s) d'exploitation sur lesquels votre serveur a été testé, voir
    \refBesoin{bnf:serverCompatibility}.
  \item Navigateur(s) web sur lesquels votre client a été testé incluant la
    version de celui-ci, voir \refBesoin{bnf:clientCompatibility}.
    \item Instructions pour configurer et lancer le serveur.
    \item Instructions pour installer et exécuter le client, y compris la configuration des dépendances (par exemple, via \verb!npm install!).
  \end{itemize}
  
}
{}
{}

\besoin{bnf:codeStructure}
{\textcolor{red}{Structuration du code}}
{
  Le code de l'application est organisé de manière claire et modulaire pour faciliter la maintenance et l'extensibilité :
  \begin{itemize}
    \item \textbf{Frontend} : Les composants Vue.js sont organisées dans le dossier \verb!frontend/src/components!, avec des fichiers comme \verb!Edit.vue! (édition d'images),
     \verb!Gallery.vue! (affichage des galeries), \verb!Login.vue! (authentification), etc.
      Les fichiers TypeScript pour la logique API est placés dans \verb!frontend/src!, comme \verb!http-api.ts! pour les appels au serveur. Les traductions sont stockées dans \verb!frontend/src/locale! (les fichiers \verb!fr.json!, \verb!en.json!, \verb!ar.json!).
    \item \textbf{Backend} : Les classes Java sont organisées dans \verb!backend/src/main/java/pdl/backend! avec des sous-dossiers comme \verb!Image! (gestion des images, comme, \verb!ImageController.java!, \verb!ImageDao.java!), \verb!Processing! (traitement d'images, par exemple, \verb!Traitement.java!), \verb!User! (gestion des utilisateurs, les fichiers, \verb!UserController.java!), et \verb!Security! (authentification, par exemple, \verb!PasswordService.java!). Les ressources (comme les images) sont stockées dans \verb!backend/src/main/resources/images!.
  \end{itemize}
}
{}
{}

\besoin{bnf:doc}
{\textcolor{red}{Documentation Backend et Frontend}}
{
  Le backend sera documenté à l'aide de Javadoc, le frontend avec Typedoc, et sera
  accessible dans le répertoire docs.
  }
{}
{}

\end{document}


